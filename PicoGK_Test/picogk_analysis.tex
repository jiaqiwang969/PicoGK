\documentclass[12pt,a4paper]{article}

% 包导入
\usepackage{xeCJK}
\usepackage{fontspec}
\usepackage{amsmath}
\usepackage{amssymb}
\usepackage{graphicx}
\usepackage{listings}
\usepackage{xcolor}
\usepackage{hyperref}
\usepackage{geometry}
\usepackage{fancyhdr}
\usepackage{tocloft}
\usepackage{titlesec}
\usepackage{enumitem}

% 页面设置
\geometry{left=2.5cm,right=2.5cm,top=3cm,bottom=3cm}

% 中文字体设置
\setCJKmainfont{STSong}
\setCJKsansfont{STHeiti}
\setCJKmonofont{STFangsong}

% 代码样式设置
\lstset{
    basicstyle=\ttfamily\small,
    keywordstyle=\color{blue}\bfseries,
    commentstyle=\color{gray}\itshape,
    stringstyle=\color{red},
    numbers=left,
    numberstyle=\tiny\color{gray},
    stepnumber=1,
    numbersep=8pt,
    backgroundcolor=\color{white},
    showspaces=false,
    showstringspaces=false,
    showtabs=false,
    frame=single,
    tabsize=4,
    captionpos=b,
    breaklines=true,
    breakatwhitespace=false,
    escapeinside={\%*}{*)},
    xleftmargin=2em,
    xrightmargin=2em,
    aboveskip=1em
}

% C# 语言定义
\lstdefinelanguage{CSharp}{
    morekeywords={abstract,as,base,bool,break,byte,case,catch,char,checked,class,const,continue,decimal,default,delegate,do,double,else,enum,event,explicit,extern,false,finally,fixed,float,for,foreach,goto,if,implicit,in,int,interface,internal,is,lock,long,namespace,new,null,object,operator,out,override,params,private,protected,public,readonly,ref,return,sbyte,sealed,short,sizeof,stackalloc,static,string,struct,switch,this,throw,true,try,typeof,uint,ulong,unchecked,unsafe,ushort,using,virtual,void,volatile,while},
    sensitive=true,
    morecomment=[l]{//},
    morecomment=[s]{/*}{*/},
    morestring=[b]",
}

% 超链接设置
\hypersetup{
    colorlinks=true,
    linkcolor=blue,
    filecolor=magenta,
    urlcolor=cyan,
    citecolor=green,
}

% 页眉页脚
\pagestyle{fancy}
\fancyhf{}
\fancyhead[L]{PicoGK 第一性原理分析}
\fancyhead[R]{\thepage}
\fancyfoot[C]{LEAP 71 - Computational Engineering}

% 标题格式
\titleformat{\section}{\Large\bfseries}{\thesection}{1em}{}
\titleformat{\subsection}{\large\bfseries}{\thesubsection}{1em}{}
\titleformat{\subsubsection}{\normalsize\bfseries}{\thesubsubsection}{1em}{}

\begin{document}

% 标题页
\begin{titlepage}
    \centering
    \vspace*{2cm}

    {\Huge\bfseries PicoGK 几何内核\\[0.5cm]第一性原理分析\par}

    \vspace{1.5cm}

    {\Large 从体素到计算工程的完整技术剖析\par}

    \vspace{2cm}

    {\large
    \textbf{作者:} Claude Opus 4.5\\[0.3cm]
    \textbf{日期:} \today\\[0.3cm]
    \textbf{版本:} 1.0\\[0.3cm]
    }

    \vfill

    {\large
    基于 PicoGK 1.7.7.5\\
    LEAP 71 - Computational Engineering\\
    \url{https://picogk.org}
    \par}

    \vspace{1cm}
\end{titlepage}

% 摘要
\begin{abstract}
\noindent
本文从第一性原理出发,深入分析 LEAP 71 开发的 PicoGK 几何内核的实现原理。PicoGK 是一个基于体素(Voxel)和有符号距离场(Signed Distance Field, SDF)的计算几何库,专为计算工程(Computational Engineering)设计。

本文将系统性地剖析 PicoGK 的核心概念、架构设计、数学基础、实现细节以及工程应用,涵盖从底层 OpenVDB 数据结构到高层 C\# API 的完整技术栈。通过理论分析与实际代码示例相结合,本文旨在为读者提供对现代计算几何系统的深刻理解。

\textbf{关键词:} 体素几何、有符号距离场、OpenVDB、计算工程、隐式曲面、布尔运算、C\#/C++ 互操作
\end{abstract}

\newpage

% 目录
\tableofcontents
\newpage

% 正文开始
\section{引言}

\subsection{背景与动机}

在传统的计算机辅助设计(CAD)系统中,几何体通常使用边界表示(B-Rep, Boundary Representation)或构造实体几何(CSG, Constructive Solid Geometry)来描述。这些方法在处理复杂的拓扑变化、布尔运算和自由形态建模时往往面临数值稳定性和性能问题。

PicoGK 采用了一种根本不同的方法:\textbf{基于体素的隐式几何表示}。这种方法将三维空间离散化为规则的体素网格,每个体素存储一个有符号距离值,表示该点到最近表面的距离。这种表示方法具有以下优势:

\begin{itemize}[leftmargin=2em]
    \item \textbf{拓扑鲁棒性:} 布尔运算和形态学操作在体素空间中变得简单且数值稳定
    \item \textbf{自然支持复杂几何:} 可以轻松表示任意拓扑的几何体,包括多连通域
    \item \textbf{统一的数据结构:} 所有几何操作都在同一数据结构上进行
    \item \textbf{并行计算友好:} 体素操作天然适合并行化
    \item \textbf{适合增材制造:} 直接对应 3D 打印的层切片过程
\end{itemize}

\subsection{PicoGK 的设计哲学}

PicoGK 的名称来源于 "Pico"(微小)和 "Geometry Kernel"(几何内核),体现了其\textbf{精简指令集}的设计哲学。与传统 CAD 系统提供数百个建模工具不同,PicoGK 只提供少量核心操作:

\begin{enumerate}[leftmargin=2em]
    \item \textbf{创建原语:} 球体、晶格、网格、隐式函数
    \item \textbf{布尔运算:} 并集、差集、交集
    \item \textbf{形态学操作:} 偏移、平滑、壳体
    \item \textbf{转换操作:} 体素 $\leftrightarrow$ 网格 $\leftrightarrow$ 标量场
\end{enumerate}

这种精简设计使得 PicoGK 易于学习和使用,同时通过组合这些基本操作可以创建极其复杂的几何体。这与 RISC(精简指令集计算机)的设计理念相似。

\subsection{技术栈概览}

PicoGK 采用分层架构:

\begin{itemize}[leftmargin=2em]
    \item \textbf{底层:} OpenVDB(C++)- 高性能稀疏体素数据结构
    \item \textbf{中间层:} PicoGK Runtime(C++)- 几何算法实现
    \item \textbf{上层:} PicoGK C\# API - 用户友好的接口
\end{itemize}

这种设计既保证了性能(C++ 底层),又提供了生产力(C\# 上层)。

\newpage

\section{第一性原理:核心数学概念}

\subsection{体素(Voxel)}

\subsubsection{定义}

体素是三维空间中的体积元素(Volume Element),类似于二维图像中的像素(Pixel)。在 PicoGK 中,三维空间被离散化为规则的立方体网格:

\begin{equation}
\mathbb{R}^3 \rightarrow \mathbb{Z}^3, \quad (x, y, z) \mapsto \left(\left\lfloor\frac{x}{\Delta}\right\rfloor, \left\lfloor\frac{y}{\Delta}\right\rfloor, \left\lfloor\frac{z}{\Delta}\right\rfloor\right)
\end{equation}

其中 $\Delta$ 是体素尺寸(voxel size),在 PicoGK 中通常为 0.1mm 到 1.0mm。

\subsubsection{体素表示的优势}

\begin{enumerate}[leftmargin=2em]
    \item \textbf{离散化简化:} 连续的几何问题转化为离散的网格问题
    \item \textbf{空间查询高效:} $O(1)$ 时间复杂度的点查询
    \item \textbf{布尔运算简单:} 逐体素的逻辑运算
\end{enumerate}

\subsection{有符号距离场(Signed Distance Field, SDF)}

\subsubsection{数学定义}

有符号距离场是一个标量函数 $\phi: \mathbb{R}^3 \rightarrow \mathbb{R}$,定义为:

\begin{equation}
\phi(\mathbf{x}) = \begin{cases}
-d(\mathbf{x}, \partial\Omega) & \text{if } \mathbf{x} \in \Omega \\
0 & \text{if } \mathbf{x} \in \partial\Omega \\
+d(\mathbf{x}, \partial\Omega) & \text{if } \mathbf{x} \notin \Omega
\end{cases}
\end{equation}

其中:
\begin{itemize}[leftmargin=2em]
    \item $\Omega$ 是几何体的内部
    \item $\partial\Omega$ 是几何体的表面
    \item $d(\mathbf{x}, \partial\Omega) = \min_{\mathbf{y} \in \partial\Omega} \|\mathbf{x} - \mathbf{y}\|$ 是点到表面的最短距离
\end{itemize}

\subsubsection{SDF 的性质}

\begin{enumerate}[leftmargin=2em]
    \item \textbf{Lipschitz 连续:} $|\phi(\mathbf{x}) - \phi(\mathbf{y})| \leq \|\mathbf{x} - \mathbf{y}\|$
    \item \textbf{梯度归一化:} $\|\nabla\phi(\mathbf{x})\| = 1$ (在表面附近)
    \item \textbf{零等值面:} 表面定义为 $\{\mathbf{x} : \phi(\mathbf{x}) = 0\}$
\end{enumerate}

\subsection{隐式曲面(Implicit Surface)}

\subsubsection{定义}

隐式曲面通过函数 $f: \mathbb{R}^3 \rightarrow \mathbb{R}$ 的零等值面定义:

\begin{equation}
S = \{\mathbf{x} \in \mathbb{R}^3 : f(\mathbf{x}) = 0\}
\end{equation}

在 PicoGK 中,隐式函数通过 \texttt{IImplicit} 接口实现:

\begin{lstlisting}[language=CSharp,caption={隐式函数接口}]
public interface IImplicit
{
    float fSignedDistance(in Vector3 vec);
}
\end{lstlisting}

\subsubsection{常见隐式曲面}

\textbf{1. 球体:}
\begin{equation}
f(\mathbf{x}) = \|\mathbf{x} - \mathbf{c}\| - r
\end{equation}

\textbf{2. Gyroid 三周期极小曲面:}
\begin{equation}
f(x, y, z) = \sin(x)\cos(y) + \sin(y)\cos(z) + \sin(z)\cos(x)
\end{equation}

\textbf{3. 圆环(Torus):}
\begin{equation}
f(x, y, z) = \left(\sqrt{x^2 + y^2} - R\right)^2 + z^2 - r^2
\end{equation}

\newpage

\section{OpenVDB:稀疏体素数据结构}

\subsection{为什么需要稀疏表示?}

对于高分辨率的体素网格,密集存储是不可行的。例如,一个 $1000 \times 1000 \times 1000$ 的网格需要 10 亿个体素。如果每个体素存储 4 字节的浮点数,总共需要 4GB 内存。

然而,在实际应用中,大部分体素是空的或具有相同的值。OpenVDB 利用这一特性,使用\textbf{稀疏数据结构}来高效存储体素数据。

\subsection{OpenVDB 的 B+ 树结构}

OpenVDB 使用分层的 B+ 树结构来存储体素数据:

\begin{itemize}[leftmargin=2em]
    \item \textbf{Root Node:} 根节点,存储指向子节点的指针
    \item \textbf{Internal Nodes:} 内部节点,多层级的索引结构
    \item \textbf{Leaf Nodes:} 叶节点,存储实际的体素值(通常 $8^3 = 512$ 个体素)
\end{itemize}

\subsection{空间复杂度分析}

对于表面附近的窄带(narrow band)表示:

\begin{equation}
\text{Memory} = O(N \cdot W)
\end{equation}

其中:
\begin{itemize}[leftmargin=2em]
    \item $N$ 是表面面积(以体素为单位)
    \item $W$ 是窄带宽度(通常 3-5 个体素)
\end{itemize}

这使得内存使用量与表面积成正比,而不是体积成正比。

\subsection{时间复杂度}

\begin{itemize}[leftmargin=2em]
    \item \textbf{点查询:} $O(\log N)$ 或接近 $O(1)$(通过缓存)
    \item \textbf{遍历:} $O(K)$,其中 $K$ 是活跃体素数量
    \item \textbf{布尔运算:} $O(K_1 + K_2)$
\end{itemize}

\newpage

\section{PicoGK 架构设计}

\subsection{分层架构}

\begin{verbatim}
┌─────────────────────────────────────────┐
│     C# Application Layer                │
│  (用户代码,高层几何操作)                │
└─────────────────────────────────────────┘
                  ↓
┌─────────────────────────────────────────┐
│     PicoGK C# API                       │
│  (Voxels, Mesh, Lattice, ScalarField)   │
└─────────────────────────────────────────┘
                  ↓ P/Invoke
┌─────────────────────────────────────────┐
│     PicoGK Runtime (C++)                │
│  (几何算法,布尔运算,网格化)            │
└─────────────────────────────────────────┘
                  ↓
┌─────────────────────────────────────────┐
│     OpenVDB (C++)                       │
│  (稀疏体素数据结构,I/O)                │
└─────────────────────────────────────────┘
\end{verbatim}

\subsection{核心类设计}

\subsubsection{Voxels 类}

\texttt{Voxels} 是 PicoGK 的核心类,封装了 OpenVDB 的体素场:

\begin{lstlisting}[language=CSharp,caption={Voxels 类结构}]
public partial class Voxels : IDisposable
{
    private IntPtr m_hThis;  // C++ 对象句柄

    // 构造函数
    public Voxels();
    public Voxels(in Mesh msh);
    public Voxels(in Lattice lat);
    public Voxels(in IImplicit impl, in BBox3 bounds);

    // 布尔运算
    public void BoolAdd(in Voxels operand);
    public void BoolSubtract(in Voxels operand);
    public void BoolIntersect(in Voxels operand);

    // 形态学操作
    public void Offset(float distMM);
    public void Smoothen(float distMM);
    public Voxels voxShell(float offset);

    // 转换
    public Mesh mshAsMesh();
}
\end{lstlisting}

\end{document}
